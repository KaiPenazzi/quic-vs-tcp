\subsection{Build und Ausführung}
Der Server kann problemlos direkt über Gradle mit dem Task \texttt{run} gestartet werden:

\begin{lstlisting}
./gradlew server:run
\end{lstlisting}

Alternativ kann der Server als ausführbare Fat JAR gebaut werden. Hierzu wird das Gradle-Plugin \texttt{com.github.gooogler.shadow} verwendet, das sämtliche Abhängigkeiten in eine einzelne ausführbare JAR integriert.

\begin{lstlisting}
./gradlew shadowJar
\end{lstlisting}

Anschließend kann der Server unabhängig von Gradle direkt über die Java-Laufzeitumgebung gestartet werden:

\begin{lstlisting}
java -jar server/build/libs/server-all.jar
\end{lstlisting}
