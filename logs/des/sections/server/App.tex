\subsection{App}

Die Klasse \texttt{App} dient als Einstiegspunkt der Serveranwendung und übernimmt die Initialisierung des QUIC-Servers.

Zu Beginn wird der zuvor erzeugte PKCS12-Keystore geladen.

\begin{lstlisting}
String KEYSTORE_PATH = "../keystore.p12";
String KEYSTORE_PASSWORD = "keystorepass";
String KEY_ALIAS = "kwikserver";

KeyStore keystore = KeyStore.getInstance("PKCS12");

try (FileInputStream keystoreStream = new FileInputStream(new File(KEYSTORE_PATH).getAbsolutePath())) {
    keystore.load(keystoreStream, KEYSTORE_PASSWORD.toCharArray());
}
\end{lstlisting}

Anschließend wird ein \texttt{Logger} konfiguriert, der Laufzeitinformationen wie Verbindungsereignisse, Stream-Daten und Warnungen ausgibt. Ohne einen Logger kann der Server nicht gestartet werden.

\begin{lstlisting}
Logger log = new SysOutLogger();

log.timeFormat(Logger.TimeFormat.Long);
log.logWarning(true);
log.logInfo(true);
log.logStream(true);
\end{lstlisting}

Die unterstützten QUIC-Versionen werden explizit festgelegt. In diesem Demonstrator wird QUIC Version~1 verwendet.

\begin{lstlisting}
List<QuicConnection.QuicVersion> supportedVersions = new ArrayList<>();
supportedVersions.add(QuicConnection.QuicVersion.V1);
\end{lstlisting}

Über das \texttt{ServerConnectionConfig}-Objekt werden Verbindungsparameter konfiguriert, darunter Zeitlimits, maximale Stream-Anzahlen sowie die Verwendung von Retry-Paketen.

\begin{lstlisting}
ServerConnectionConfig config = ServerConnectionConfig.builder()
        .maxIdleTimeoutInSeconds(5000)
        .maxOpenPeerInitiatedUnidirectionalStreams(10)
        .maxOpenPeerInitiatedBidirectionalStreams(100)
        .retryRequired(true)
        .connectionIdLength(8)
        .build();
\end{lstlisting}

Der eigentliche QUIC-Server wird anschließend über den \texttt{ServerConnector} erstellt.
Dabei werden unter anderem der Serverport, die unterstützten Protokollversionen, das Schlüsselmaterial sowie der Logger registriert.

\begin{lstlisting}
ServerConnector connector = ServerConnector.builder()
        .withPort(7000)
        .withSupportedVersions(supportedVersions)
        .withKeyStore(keystore, KEY_ALIAS, KEYSTORE_PASSWORD.toCharArray())
        .withConfiguration(config)
        .withLogger(log)
        .build();
\end{lstlisting}

Über \texttt{registerApplicationProtocol} wird schließlich das Anwendungsprotokoll \texttt{quic} mit der zuvor definierten \texttt{MyQuicFactory} verknüpft.

\begin{lstlisting}
connector.registerApplicationProtocol("quic", new MyQuicFactory(log));
\end{lstlisting}

Mit dem Aufruf von \texttt{connector.start()} wird der Server gestartet.

\begin{lstlisting}
connector.start();
\end{lstlisting}
