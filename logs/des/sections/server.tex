\section{Server}
\subsection{Schlüsselgenerierung}

Obwohl der Client im Demonstrator keine Serverauthentifizierung durchführt, benötigt QUIC aufgrund der integrierten TLS~1.3-Verschlüsselung dennoch entsprechendes Schlüsselmaterial.
Das folgende Bash-Skript wurde von ChatGPT erstellt und erzeugt das Schlüsselmaterial.

\begin{verbatim}
#!/bin/bash

# Generiert von ChatGPT
# 1️⃣ Erstelle ein Self-Signed Zertifikat mit SAN
openssl req -x509 -nodes -days 365 \
  -newkey rsa:2048 \
  -keyout server.key \
  -out server.crt \
  -subj "/CN=localhost" \
  -addext "subjectAltName=DNS:localhost,IP:127.0.0.1"

# 2️⃣ Exportiere in PKCS12-Keystore mit Alias 'kwikserver'
openssl pkcs12 -export \
  -in server.crt \
  -inkey server.key \
  -name kwikserver \
  -out keystore.p12 \
  -passout pass:keystorepass

# 3️⃣ Optional: Prüfe Inhalt des Keystores
echo "Folgende Einträge befinden sich im Keystore:"
keytool -list -keystore keystore.p12 -storetype PKCS12 -storepass keystorepass
\end{verbatim}

\subsection{GlobalState}

Für den Demonstrator wird ein globaler Zustand implementiert, der von allen aktiven QUIC-Verbindungen gemeinsam genutzt wird. 
Dies simuliert in einer realen Anwendung eine einfache Persistenz, auf die mehrere Verbindungen zugreifen können. 
In diesem Fall ist der globale Zustand nur ein einfacher String, der gelesen und geschrieben werden kann.

\begin{verbatim}
public class MyGlobelState {
    private String state = "";

    public String getState() {
        return this.state;
    }

    public void setState(String newState) {
        this.state = newState;
    }
}
\end{verbatim}

\subsection{MyConnectionListener}

\texttt{MyConnectionListener} implementiert die Schnittstelle \texttt{ConnectionListener} und überwacht wichtige Ereignisse der QUIC-Verbindung.
Die Überwachung beschränkt sich dabei auf den Aufbau und die Beendigung der Verbindung, da das Interface keine weiteren Methoden bereitstellt.

\begin{lstlisting}
public class MyConnectionListener implements ConnectionListener {
    private Logger log;

    public MyConnectionListener(Logger logger) {
        this.log = logger;
    }

    @Override
    public void connected(ConnectionEstablishedEvent event) {
        log.info("connection established");
    }

    @Override
    public void disconnected(ConnectionTerminatedEvent event) {
        log.info("connection terminated: " + event.errorDescription());
    }
}
\end{lstlisting}

\subsection{MyQuicConnection}

Die Klasse \texttt{MyQuicConnection} stellt die zentrale Verbindungskomponente des Servers dar. 
Sie implementiert die Schnittstelle \texttt{ApplicationProtocolConnection} des Kwik-Frameworks und wird für jede eingehende QUIC-Verbindung instanziiert.
Im Konstruktor wird ein \texttt{MyConnectionListener} gesetzt. Die \texttt{connected}-Methode des Listeners wird dabei allerdings nicht aufgerufen, da die Verbindung bereits vorher aufgebaut wurde.

Die Hauptaufgaben dieser Klasse sind:

\begin{itemize}
    \item Verwaltung des lokalen Verbindungszustands (\texttt{ConnectionState}) und Zugriff auf den globalen Zustand (\texttt{MyGlobalState}).
    \item Akzeptieren von von Peers initiierten Streams über \texttt{acceptPeerInitiatedStream}.
    \item Unterscheidung zwischen unidirektionalen und bidirektionalen Streams.
    \item Verarbeitung von Nachrichten aus bidirektionalen Streams, inklusive Befehlen zur Zustandsabfrage oder -änderung:
    \begin{itemize}
        \item \texttt{get / set} für den lokalen Verbindungszustand
        \item \texttt{gget / gset} für den globalen Zustand
        \item Alle anderen Nachrichten werden einfach zurückgesendet (Echo)
    \end{itemize}
    \item Versand von Nachrichten über die bidirektionalen Streams.
\end{itemize}

Die Verarbeitung jedes Streams erfolgt in einem eigenen Thread, um parallele Kommunikation zu ermöglichen.


\begin{lstlisting}
public class MyQuicConnection implements ApplicationProtocolConnection {
    private final QuicConnection quicConnection;
    private final Logger log;
    private String ConnectionState;
    private MyGlobelState globelState;

    public MyQuicConnection(QuicConnection quicConnection, Logger logger, MyGlobelState state) {
        this.quicConnection = quicConnection;
        this.quicConnection.setConnectionListener(new MyConnectionListener(logger));
        this.log = logger;
        this.ConnectionState = "";
        this.globelState = state;
    }

    @Override
    public void acceptPeerInitiatedStream(QuicStream stream) {
        new Thread(() -> handleClient(stream)).start();
    }

    private void handleClient(QuicStream stream) {
        if (stream.isUnidirectional()) {
            try (InputStream in = stream.getInputStream()) {

                byte[] buf = new byte[4096];
                int len;

                while ((len = in.read(buf)) != -1) {
                    if (len == 0)
                        continue;

                    String msg = new String(buf, 0, len, StandardCharsets.UTF_8);
                    log.info(this.hashCode() + ": stream ID: " + stream.getStreamId() + ": " + msg);
                }

            } catch (Exception e) {
                log.error("stream failed", e);
            }
        }
        if (stream.isBidirectional()) {
            try (InputStream in = stream.getInputStream()) {

                byte[] buf = in.readAllBytes();
                String msg = new String(buf, StandardCharsets.UTF_8);
                log.info(this.hashCode() + ": stream ID: " + stream.getStreamId() + ": " + msg);

                String[] parts = msg.split(" ");

                switch (parts[0]) {
                    case "get":
                        this.send("state: " + this.ConnectionState, stream);
                        break;

                    case "set":
                        try {
                            this.ConnectionState = parts[1];
                            this.send("state: " + this.ConnectionState, stream);
                        } catch (IndexOutOfBoundsException e) {
                            this.send("value is missing: 'set <value>' ", stream);
                        }
                        break;
                    case "gget":
                        this.send("global state: " + this.globelState.getState(), stream);
                        break;

                    case "gset":
                        try {
                            this.globelState.setState(parts[1]);
                            this.send("global state: " + this.globelState.getState(), stream);
                        } catch (IndexOutOfBoundsException e) {
                            this.send("value is missing: 'gset <value>' ", stream);
                        }
                        break;
                    default:
                        this.send("echo: " + msg, stream);
                }

            } catch (Exception e) {
                log.error("stream failed", e);
            }
        }

    }

    private void send(String msg, QuicStream stream) {
        try (OutputStream out = stream.getOutputStream()) {
            out.write(msg.getBytes());
            out.close();
        } catch (Exception e) {
            log.error("could not send msg: ", e);
        }

    }
}
\end{lstlisting}



