\subsection{Commands}

Die Client-Anwendung nutzt Picocli, um die Benutzereingaben als Subcommands bereitzustellen. Jeder Command implementiert entweder das \texttt{Callable}- oder das \texttt{Runnable}-Interface und kann eigene Parameter definieren, die Picocli automatisch aus den eingegebenen Strings parsed.

Ein Beispiel hierfür ist der \texttt{OpenStream}-Command.
Dieser erwartet als Parameter die \texttt{connection\_id}, die angibt, über welche bestehende QUIC-Verbindung ein neuer Stream geöffnet werden soll.
Innerhalb des Commands greift die Implementierung auf \texttt{MyConnectionHandler} zu um einen neuen Stream zu eröffnen.
Auf diese Weise abstrahiert jeder Command die direkte Interaktion mit den QUIC-Verbindungen und ermöglicht eine übersichtliche, modulare Steuerung des Clients.


\begin{lstlisting}
@Command(name = "open_stream", description = "open a stream")
public class OpenStream implements Callable<Integer> {
    private final MyConnectionHandler handler;

    @Parameters(index = "0", description = "Connection ID")
    public int connection_id;

    public OpenStream(MyConnectionHandler handler) {
        this.handler = handler;
    }

    @Override
    public Integer call() throws IOException {
        return this.handler.getConnection(this.connection_id).open_stream();
    }
}
\end{lstlisting}
