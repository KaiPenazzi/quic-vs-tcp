\subsection{Build und Ausführung}

Da der Client als Kommandozeilenwerkzeug implementiert ist und kontinuierlich Benutzereingaben über \texttt{System.in} verarbeitet, kommt es zu Fehlern bei der ausführung mit \texttt{gradlew client:run}.

Aus diesem Grund wird der Client als ausführbare JAR-Datei bereitgestellt.
Hierfür wird das Gradle-Plugin \texttt{com.github.gooogler.shadow} verwendet, das eine "Fat JAR" erzeugt.
Dabei werden sämtliche Abhängigkeiten, in eine einzelne ausführbare Datei integriert.

Die JAR-Datei kann direkt über die Java-Laufzeitumgebung gestartet werden.

\begin{lstlisting}
#!/bin/bash

./gradlew client:shadowJar
java -jar client/build/libs/client-all.jar
\end{lstlisting}
