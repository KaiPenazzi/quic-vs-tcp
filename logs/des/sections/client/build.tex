\subsection{Build und Ausführung}

Da der Client als Kommandozeilenwerkzeug implementiert ist und kontinuierlich Benutzereingaben über \texttt{System.in} verarbeitet, kommt es zu Fehlern bei der ausführung mit \texttt{gradlew client:run}.

Aus diesem Grund muss der Client als ausführbare JAR-Datei gestartet werden.
Hierfür wird ebenfalls das Gradle-Plugin \texttt{com.github.gooogler.shadow} verwendet.

\begin{lstlisting}
./gradlew client:shadowJar
java -jar client/build/libs/client-all.jar
\end{lstlisting}
