\subsection{App}

Die Klasse \texttt{App} dient als Kommandozeilen-Interface für den QUIC-Client-Demonstrator. 
Sie nutzt \texttt{picocli} zur Verarbeitung von Subcommands.

Die Verwaltung der QUIC-Verbindungen erfolgt über \texttt{MyConnectionHandler}.
Benutzereingaben werden in einer Endlosschleife eingelesen und anschließend an die registrierten Subcommands weitergeleitet. 
Die Zerlegung des Eingabestrings in einzelne Argumente erfolgt über die Methode \texttt{splitArgs}, die von dem KI-Modell Gemini generiert wurde und dafür sorgt, dass auch Argumente in Anführungszeichen korrekt als ein String behandelt werden.

\begin{lstlisting}
@Command(name = "", subcommands = {})
public class App {
    public static void main(String[] args) throws Exception {
        MyConnectionHandler handler = new MyConnectionHandler();

        Scanner scanner = new Scanner(System.in);

        CommandLine cmd = new CommandLine(new App());
        cmd.addSubcommand("connect", new Connect(handler));
        cmd.addSubcommand("disconnect", new Disconnect(handler));
        cmd.addSubcommand("openStream", new OpenStream(handler));
        cmd.addSubcommand("closeStream", new CloseStream(handler));
        cmd.addSubcommand("sendOver", new SendOver(handler));
        cmd.addSubcommand("getState", new GetState(handler));
        cmd.addSubcommand("setState", new SetState(handler));
        cmd.addSubcommand("stats", new Stats(handler));
        cmd.addSubcommand("setGState", new SetGlobalState(handler));
        cmd.addSubcommand("getGState", new GetGlobalState(handler));

        while (true) {
            System.out.print("> ");
            String input = scanner.nextLine();

            if (input.equals("exit"))
                break;

            String[] parsedArgs = splitArgs(input);
            cmd.execute(parsedArgs);
        }

        handler.closeAll();
    }

    // von Gemini generierte Methode zum Parsen von Anführungszeichen
    public static String[] splitArgs(String string) {
        List<String> matchList = new ArrayList<String>();
        Pattern regex = Pattern.compile("[^\\s\"]+|\"([^\"]*)\"");
        Matcher regexMatcher = regex.matcher(string);
        while (regexMatcher.find()) {
            if (regexMatcher.group(1) != null)
                matchList.add(regexMatcher.group(1));
            else
                matchList.add(regexMatcher.group());
        }
        return matchList.toArray(new String[0]);
    }
}
\end{lstlisting}
