\section{Einleitung}

Im Internet werden Nachrichten hauptsächlich über die Transportproto\-kolle TCP und UDP übertragen. TCP gewährleistet eine zuverlässige, geordnete und verlustfreie Übertragung, kann aber durch seinen aufwendigen Verbindungsaufbau, eingeschränkte Erweiterungsmöglichkeiten und begrenzte integrierte Sicherheitsmechanismen in modernen, latenz- und sicherheitskritischen Anwendungen an seine Grenzen stoßen. Das QUIC-Protokoll wurde entwickelt, um diese Limitierungen zu überwinden und moderne Anforderungen wie geringe Latenz, zuverlässige Übertragung und integrierte Verschlüsselung zu erfüllen.
\\
Diese Ausarbeitung gibt zunächst eine Einführung in die grundlegenden Konzepte und Eigenschaften des QUIC-Protokolls.
Dabei werden zentrale Mechanismen wie Verbindungsaufbau, Multiplexing und integrierte Verschlüsselung erläutert.
Darauf aufbauend werden anschließend die drei in Java verwendbaren QUIC-Frameworks Kwik, Quiche4J und Netty mitein\-ander verglichen.
Der Vergleich betrachtet unter anderem den Funktionsumfang, den Abstraktionsgrad sowie die Einsatzgebiete, für die sich die jeweiligen Frameworks am besten eignen.
\\
Im letzten Teil der Arbeit wird eine eigene Implementierung vorgestellt, die auf dem Framework Kwik basiert.
Dabei werden Designentscheidungen, die konkrete Umsetzung sowie gemachte Erfahrungen und Herausforderungen beschrieben.
Ziel ist es, sowohl einen praxisnahen Einblick in die Arbeit mit QUIC in Java zu geben als auch die Eignung von Kwik für reale Anwendungsfälle zu evaluieren.
\newpage
