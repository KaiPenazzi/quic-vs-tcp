\subsection{State- und Event-Transparenz}

Obwohl QUIC vollständig im Userspace implementiert ist, gewähren die betrachteten Java-Frameworks nur begrenzte Einblicke in den aktuellen internen Zustand einer Verbindung.
Ein direkter Zugriff oder eine Manipulation der QUIC-State-Machine ist in der Regel nicht vorgesehen.
Stattdessen ähneln sich die Frameworks in den Abstraktionen, die sie dem Entwickler zur Verfügung stellen.

Alle drei Implementierungen stellen Objekte zur Repräsentation von Verbindungen und Streams bereit und ermöglichen es, über Callback-Mechanismen oder Event-Handler über den erfolgreichen Aufbau oder das Beenden einer Verbindung.
Darüber hinaus bieten die Frameworks Zugriff auf Statistik-Objekte, die grundlegende Metriken zur Überwachung der Verbindung enthalten.

Diese Statistiken umfassen typischerweise Informationen über gesendete und empfangene Daten beispielsweise:

\begin{itemize}
    \item empfangene Pakete
    \item gesendete Pakete
    \item verlorene Pakete
    \item geschätzte Round-Trip-Time
    \item aktuelles Congestion Window
    \item geschätzte Übertragungsrate
\end{itemize}

Damit ermöglichen die Frameworks eine Beobachtung der Übertragungsqualität, ohne jedoch eine detaillierte Kontrolle über die internen Protokollzustände offenzulegen.
Diese Designentscheidung vereinfacht die Nutzung der APIs, schränkt jedoch die Möglichkeiten für tiefgreifende Protokollmanipulationen ein.
