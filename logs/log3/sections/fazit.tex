\section{Fazit}

\subsection{Java-Frameworks}

Die drei verglichenen Java-QUIC-Implementierungen zeigen unterschiedliche Abstraktionsgrade und Integrationsaufwände:

\begin{itemize}
    \item \textbf{Kwik:} Bietet eine High-Level-API, die Handshake, Socket-Management und Stream-Verwaltung automatisch übernimmt. Ideal für schnelle Implementierungen ohne tiefe Netzwerkkenntnisse.
    \item \textbf{Quiche/Quiche4j:} Stellt lediglich die QUIC-State-Machine bereit. Der Entwickler muss UDP-Socket, Eventloop und Handshake selbst implementieren. Maximale Flexibilität, aber hoher Aufwand.
    \item \textbf{Netty QUIC:} Integriert QUIC in das Netty-Eventloop-Modell. Handshake und UDP-Transport werden größtenteils verwaltet, Streams werden über Netty-Channels bereitgestellt. Gut geeignet für Anwendungen, die bereits auf Netty basieren.
\end{itemize}

Zusammenfassend lässt sich festhalten, dass Kwik sich für übersichtliche Client- und Serveranwendungen eignet, bei denen eine klare QUIC-Abstraktion und eine einfache Stream-Verarbeitung im Vordergrund stehen.
Netty QUIC eignet sich besonders für Anwendungen, die bereits im Netty-Ökosystem umgesetzt sind oder zukünftig darauf aufbauen sollen.
Quiche4J richtet sich hingegen an Szenarien, in denen Kontrolle über den QUIC-Datenfluss erforderlich ist oder bereits eigene UDP-Datagramm- und Eventloop-Implementierungen vorhanden sind.

\subsection{QUIC}

QUIC ist ein modernes Transportprotokoll, das die grundlegenden Eigenschaften von TCP übernimmt und diese um zusätzliche Mechanismen erweitert.
Durch die Realisierung auf Basis von UDP kann QUIC ohne Anpassungen an bestehender Netzwerkinfrastruktur eingesetzt werden, da weder Router noch Endgeräte auf Kernel- oder Hardwareebene verändert werden müssen.

Ein zentrales Merkmal von QUIC ist die Implementierung im User Space.
Dadurch lassen sich neue Protokollfunktionen schneller entwickeln und ausrollen als bei klassischen Kernel-basierten Transportprotokollen.
Beispiele hierfür sind das native Stream-Multiplexing ohne Head-of-Line-Blocking sowie die enge Kopplung von Transport- und Sicherheitsmechanismen, wodurch Verschlüsselung nicht als zusätzlicher Layer, sondern als integraler Bestandteil des Protokolls umgesetzt wird.

Insgesamt zeigt sich, dass QUIC viele Einschränkungen von TCP adressiert, ohne dessen Konzepte aufzugeben. 
