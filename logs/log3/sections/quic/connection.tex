\subsection{Verbindungsaufbau}
Beim Aufbau einer Verbindung über TCP erfolgt zunächst ein dreistufiger Handshake (SYN, SYN-ACK, ACK), um eine Verbindung aufzubauen.
Dieser Handshake verursacht mindestens eine Round-Trip-Time (RTT), bevor Daten übertragen werden können.
Wird zusätzlich eine Verschlüsselung wie TLS verwendet, muss über der bereits etablierten TCP-Verbindung ein weiterer Handshake stattfinden, der in der Regel ein bis zwei zusätzliche RTTs benötigt.
In Szenarien mit vielen kurzen Verbindungen summieren sich diese Verzögerungen und kann die Performance erheblich beeinträchtigen.
\\
Das QUIC-Protokoll verwendet UDP als Transportbasis und baut darauf eine eigene Verbindungsschicht auf.
Bereits beim ersten Verbindungsaufbau werden die Verschlüsselungsschlüssel ausgehandelt, sodass dafür in der Regel eine RTT erforderlich ist.
Eine Wiederverbindung kann sogar ohne zusätzlichen Verbindungsaufbau erfolgen (0-RTT).
Durch diese Integration von Transport- und Sicherheitsmechanismen reduziert QUIC die Latenz beim Verbindungsaufbau erheblich und eignet sich besonders für Anwendungen mit häufigen oder kurzlebigen Verbindungen.
\newpage
