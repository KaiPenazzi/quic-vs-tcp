\subsection{Verbindungsaufbau}
Beim Aufbau einer Verbindung über TCP erfolgt zunächst ein dreistufiger Handshake (SYN, SYN-ACK, ACK), um eine zuverlässige, geordnete Verbindung herzustellen.
Dieser Handshake verursacht mindestens eine Round-Trip-Time (RTT), bevor Daten übertragen werden können.
Wird zusätzlich eine Verschlüsselung wie TLS eingesetzt, muss über der bereits etablierten TCP-Verbindung ein weiterer Handshake stattfinden, der in der Regel ein bis zwei zusätzliche RTTs benötigt, bis die fertige Verbindung Aufgebaut ist.
In Szenarien mit vielen kurzen Verbindungen summieren sich diese Verzögerungen und können die Performance erheblich beeinträchtigen.
\\
Das QUIC-Protokoll verwendet UDP als Transportbasis und baut darauf eine eigene Verbindungsschicht auf.
Bereits beim ersten Verbindungsaufbau werden die Verschlüsselungsschlüssel ausgehandelt, sodass dafür in der Regel eine RTT erforderlich ist.
Bei einem Wiederaufbau einer bereits bekannten Verbindung (Reconnect) kann der Handshake sogar ohne zusätzliche RTTs (0-RTT) erfolgen.
Durch diese Integration von Transport- und Sicherheitsmechanismen reduziert QUIC die Latenz beim Verbindungsaufbau erheblich und eignet sich besonders für Anwendungen mit häufigen oder kurzlebigen Verbindungen.
\newpage
