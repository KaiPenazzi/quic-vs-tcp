\subsection{Paketverlust und Fehlerbehandlung}

QUIC übernimmt viele grundlegende Konzepte der Verlustbehandlung aus TCP, darunter schnelle Wiederübertragungen sowie timeoutbasierte Mechanismen zur Verlustdetektion.
Durch die Trennung von Transport- und Anwendungsschicht sowie den Betrieb im Userspace erweitert QUIC diese Mechanismen jedoch um zusätzliche Informationen.

\subsubsection{Packet- und Datenzuordnung}

Ein zentraler Unterschied zwischen TCP und QUIC liegt in der Handhabung der Sequenznummern. Während TCP eine einzige Sequenznummer für die gesamte Übertragung verwendet, verwendet QUIC ein abgewandelteres Konzept.
Jedes Paket erhält eine monoton steigende Paketnummer, die für ACKs, RTT-Messungen und Loss Detection verwendet wird.
Die eigentlichen Daten werden zusätzlich durch Stream-ID und Stream-Offset identifiziert, wodurch die Reihenfolge innerhalb eines Streams eindeutig bestimmt ist.
Durch diese Trennung kann QUIC jederzeit genau nachvollziehen, welche Pakete bestätigt wurden und welche verloren gingen, was die RTT-Berechnung und die Verlustbehandlung deutlich vereinfacht.

\subsubsection{Ack-Zurücknehmen}

QUIC erlaubt kein Zurücknehmen bereits bestätigter Pakete. Dadurch wird die Verlustbehandlung auf beiden seiten vereinfacht. \cite{iyengar_quic_2018}

\subsubsection{Ack-Ranges}

Im Gegensatz zu TCP, das ACKs höchstens drei SACK-Ranges unterstützt, erlaubt QUIC mehrere ACK-Ranges in einem einzelnen ACK-Paket.
Diese Flexibilität beschleunigt die Paketverlust-Erkennung und reduziert unnötige Retransmissions insbesondere in Netzen mit hohem Paketverlust. \cite{iyengar_quic_2018}

\subsubsection{Korrektur für verzögerte ACKs}
QUIC-ACKs kodieren explizit die Verzögerung, die beim Empfänger zwischen dem Eintreffen eines Pakets und dem Versenden der zugehörigen Bestätigung entsteht.
Dadurch kann der Empfänger des ACKs die empfangsseitige Verzögerungen in der Schätzung der RTT berücksichtigen. \cite{iyengar_quic_2018}

\subsubsection{Fehlererkennung}

TCP verwendet auf Paketebene eine Prüfsumme, um sicherzustellen, dass Header und Daten auf dem Transportweg nicht verändert wurden.
QUIC gewährleistet die Integrität und Manipulationssicherheit seiner Pakete hingegen über die in das Protokoll integrierte TLS-1.3-Verschlüsselung.
Jedes QUIC-Paket wird dabei verschlüsselt und enthält einen Authentifizierungs-Tag, sodass Änderungen am Header oder an den enthaltenen Frames vom Empfänger zuverlässig erkannt werden.
Auf diese Weise übernimmt QUIC die Funktion der klassischen TCP-Prüfsumme, erweitert diese jedoch um kryptographisch gesicherte Integrität.
\\
Wie auch bei TCP enthält QUIC standardmäßig keine Forward Error Correction Funktionalität, aufgrund der Implementierung im Userspace ist es jedoch deutlich einfacher, solche Erweiterungen nachträglich in QUIC zu integrieren.
\newpage
