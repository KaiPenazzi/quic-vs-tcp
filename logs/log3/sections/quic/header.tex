\subsection{Paket}

QUIC-Kommunikation erfolgt über UDP-Datagramme, die ein oder mehrere QUIC-Pakete zwischen Endpunkten transportieren.
Jedes QUIC-Paket besteht aus einem Header und einem oder mehreren Frames, die die eigentlichen Nutzdaten und Steuerinformationen enthalten.
QUIC definiert dabei verschiedene Header- und Pakettypen, die unterschiedliche Funktionen wie Verbindungshandshake oder schnellen Datentransport übernehmen.

\subsubsection{Header}
QUIC läuft auf UDP, was bedeutet, dass die Zuordnung zu Anwendung und Endpunkten bereits durch die IP-Adressen und Ports auf UDP-Ebene erfolgt.
Da QUIC auf Anwendungsschicht arbeitet, muss es nur noch die Zuordnung auf Verbindungsebene sicherstellen.
QUIC verwendet zwei verschiedene Header-Typen.
Der Long Header wird für den Verbindungsaufbau verwendet, während der Short Header bei bereits etablierte Verbindungen verwendet wird.

\paragraph{Longheader:}\cite{alawaji_ietf_nodate}

\begin{itemize}
    \item \textbf{Header Form (HF):} Identifiziert den Header-Typ.
    \item \textbf{Fixed Bit (FB):} Zeigt an, ob das Paket gültig ist oder nicht. Ist das Bit auf 0 gesetzt, ist das Paket ungültig.
    \item \textbf{Long Packet Type (T):} Gibt den Typ des Long-Header-Pakets an, siehe Tabelle 1.
    \item \textbf{Type-Specific Bits (S):} Bits, die spezifisch für die jeweiligen Long-Header-Pakettypen sind.
    \item \textbf{Version ID (VID):} 32 Bit zur Identifikation der QUIC-Version.
    \item \textbf{Destination Connection ID Length (DCID Len):} Länge der Ziel-Connection-ID.
    \item \textbf{Destination Connection ID (DCID):} Ziel-Connection-ID.
    \item \textbf{Source Connection ID Length (SCID Len):} Länge der Quell-Connection-ID.
    \item \textbf{Source Connection ID (SCID):} Quell-Connection-ID.
\end{itemize}
    %zitieren nicht vergessen
\paragraph{Shortheader:} \cite{alawaji_ietf_nodate}

\begin{itemize}
    \item \textbf{Header Form (HF):} Identifiziert den Header-Typ.
    \item \textbf{Fixed Bit (FB):} Zeigt an, ob das Paket gültig ist oder nicht.
    \item \textbf{Spin Bit:} Wird zur Latenzmessung verwendet.
    \item \textbf{Reserved Bits:} Für zukünftige Erweiterungen reserviert.
    \item \textbf{Key Phase:} Gibt an, welcher Verschlüsselungsschlüssel für das Paket verwendet wird.
    \item \textbf{Packet Number Length (P):} Länge der Paketnummer.
    \item \textbf{Destination Connection ID (DCID):} Ziel-Connection-ID.
    \item \textbf{Packet Number:} Monoton steigende Paketnummer zur Verlustdetektion und ACKs.
    \item \textbf{Packet Payload:} Enthält die Frames des Pakets (STREAM, ACK, PADDING etc.).
\end{itemize}

Diese Struktur ermöglicht es QUIC, die Stärken von TCP, wie etwa zuverlässige Datenübertragung, zu übernehmen, während gleichzeitig Verschlüsselung, Integrität und eine flexible Zuordnung von Verbindungen auf Anwendungsebene implementiert werden können.
Die Verwendung unterschiedlicher Header-Typen sorgt dafür, dass das Protokoll effizient bleibt.

\subsubsection{Frame}

Ein Frame enthält die Daten eines Streams, der über QUIC übertragen wird.

\begin{itemize}
    \item \textbf{Type:} Gibt die Art des Frames an, z. B. Stream-Daten, ACK oder Control-Informationen.
    \item \textbf{Stream ID:} Identifiziert den Stream, zu dem die Daten gehören. Ungerade IDs stehen für vom Client initiierte Streams, gerade IDs für vom Server initiierte Streams.
    \item \textbf{Offset:} Gibt die Position der Daten innerhalb des Streams an, ähnlich dem TCP-Offset.
    \item \textbf{Data Length:} Die Länge der übertragenen Daten in Bytes.
    \item \textbf{Stream Data:} Die eigentlichen Nutzdaten des Streams.
\end{itemize}
