\section{Fazit}

\begin{frame}{Vergleich}
    \textbf{Kwik:}
    \begin{itemize}
        \item High-Level-API: Handshake, Socket-Management, Stream-Verwaltung automatisch
        \item Ideal für schnelle Implementierungen ohne tiefe Netzwerkkenntnisse
    \end{itemize}
    \textbf{Quiche / Quiche4J:}
    \begin{itemize}
        \item Nur die QUIC-State-Machine wird bereitgestellt
        \item Entwickler muss UDP-Socket und Eventloop selbst implementieren
        \item Maximale Flexibilität, aber hoher Implementierungsaufwand
    \end{itemize}
\end{frame}


\begin{frame}{QUIC}
\begin{itemize}
    \item TCP-Grundprinzipien werden übernommen
    \item Aufbau auf UDP:
    \begin{itemize}
        \item Keine Anpassungen an Router oder Endgeräte nötig
        \item Funktioniert ohne Kernel-/Hardwareänderungen
    \end{itemize}
    \item User-Space-Implementierung:
    \begin{itemize}
        \item Schnellere Entwicklung und Rollout neuer Protokollfunktionen
    \end{itemize}
    \item Features:
    \begin{itemize}
        \item Stream-Multiplexing
        \item Integrierte Verschlüsselung
    \end{itemize}
\end{itemize}
\end{frame}
