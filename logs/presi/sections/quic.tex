\section{QUIC}
\begin{frame}{Was ist QUIC?}
    QUIC will alles können was TCP kann nur besser
\end{frame}

\subsection{Architektur}

\begin{frame}{Was macht TCP aus?}
    \begin{itemize}
        \item Verbindungsorientiert
        \item Zuverlässige Übertragung
        \item Geordnete Zustellung
        \item Flusskontrolle
        \item Fehlerkontrolle
    \end{itemize}
\end{frame}

\begin{frame}{Grundlagen}
    \begin{itemize}
        \item Basiert auf UDP
        \item Stream Orientiert
        \item im Userspace
    \end{itemize}
    \includegraphics[width=0.8\textwidth]{pictures/quic_connection.png} \cite{kumar_quic_2020}
\end{frame}

\begin{frame}{Paket}
    \begin{itemize}
        \item Packet: \cite{alawaji_ietf_nodate} \\
            \includegraphics[width=0.8\textwidth]{pictures/quic_packet.png}
    \end{itemize}
\end{frame}

\begin{frame}{Header}
    \begin{itemize}
        \item Longheader: \cite{alawaji_ietf_nodate}
            \includegraphics[width=0.8\textwidth]{pictures/quic_header_long.png}
        \item Shortheader: \cite{alawaji_ietf_nodate}
            \includegraphics[width=0.8\textwidth]{pictures/quic_header_short.png}
        \item Header typ: \cite{alawaji_ietf_nodate}
            \includegraphics[width=0.8\textwidth]{pictures/table_header_types.png}
    \end{itemize}
\end{frame}

\subsection{Verbindung}

\begin{frame}{Verbindungsaufbau}
    \begin{figure}[h]
    \centering
        \begin{minipage}{0.48\textwidth}
            \centering
            \begin{tikzpicture}[
    scale=0.8,
    transform shape,
    node distance=1.2cm,
    every node/.style={font=\small},
    arrow/.style={->, thick},
    lifeline/.style={thick, gray}
]

\node (client) {Client};
\node[right=3cm of client] (server) {Server};

\node[below=0.5cm of client] (c1) {};
\node[below=0.5cm of server] (s1) {};
\draw[arrow] (c1) -- node[above]{SYN} (s1);

\node[below=0.8cm of c1] (c2) {};
\node[below=0.8cm of s1] (s2) {};
\draw[arrow] (s2) -- node[above]{SYN-ACK} (c2);

\node[below=0.8cm of c2] (c3) {};
\node[below=0.8cm of s2] (s3) {};
\draw[arrow] (c3) -- node[above]{ACK} (s3);

\node[below=0.8cm of c3] (c4) {};
\node[below=0.8cm of s3] (s4) {};
\draw[arrow] (c4) -- node[above]{ClientHello} (s4);

\node[below=0.8cm of c4] (c5) {};
\node[below=0.8cm of s4] (s5) {};
\draw[arrow] (s5) -- node[above]{ServerHello + Certificate} (c5);

\node[below=0.8cm of c5] (c6) {};
\node[below=0.8cm of s5] (s6) {};
\draw[arrow] (c6) -- node[above]{Client Finished} (s6);

\node[below=0.8cm of c6] (c7) {};
\node[below=0.8cm of s6] (s7) {};
\draw[arrow] (s7) -- node[above]{Server Finished} (c7);

\draw[lifeline] (client.south) -- (c7.north);
\draw[lifeline] (server.south) -- (s7.north);
\end{tikzpicture}

            \caption{TCP + TLS}
        \end{minipage}
        \hfill
        \begin{minipage}{0.48\textwidth}
            \centering
            
\begin{tikzpicture}[
    scale=0.8,
    transform shape,
    node distance=1.2cm,
    every node/.style={font=\small},
    arrow/.style={->, thick},
    lifeline/.style={thick, gray}
]

\node (client) {Client};
\node[right=3cm of client] (server) {Server};

\node[below=0.5cm of client] (c1) {};
\node[below=0.5cm of server] (s1) {};

\draw[arrow] (c1) -- node[above]{Initial} (s1);

\node[below=0.8cm of c1] (c2) {};
\node[below=0.8cm of s1] (s2) {};

\draw[arrow] (s2) -- node[above]{Initial + Handshake} (c2);

\node[below=0.8cm of c2] (c3) {};
\node[below=0.8cm of s2] (s3) {};

\draw[arrow] (c3) -- node[above]{Client Finished} (s3);

\draw[lifeline] (client.south) -- (c3.north);
\draw[lifeline] (server.south) -- (s3.north);

\end{tikzpicture}

            \caption{QUIC}
        \end{minipage}
    \end{figure}
\end{frame}

\begin{frame}{Flusskontrolle}
    \textbf{TCP:}
    \begin{itemize}
        \item Algorithmen wie NewReno oder CUBIC
        \item Überwacht die Menge an unbestätigten Paketen
        \item Bremst das Senden neuer Pakete bei Paketverlusten
    \end{itemize}

    \vspace{0.3cm}
    \textbf{QUIC:}
    \begin{itemize}
        \item Übernimmt die Grundprinzipien von TCP
        \item Verfolgt die Übertragung auf Ebene der Bytes-in-Flight
        \item Feinkörnigere Kontrolle der Daten
    \end{itemize}
\end{frame}

\subsection{Fehlerbehandlung}

\begin{frame}{Packet- und Datenzuordnung}
    \textbf{TCP:}
    \begin{itemize}
        \item Pakete mit denselben Daten teilen sich die gleiche Sequenznummer
        \item ACKs beziehen sich auf Sequenznummern, nicht direkt auf einzelne Pakete
    \end{itemize}

    \vspace{0.3cm}
    \textbf{QUIC:}
    \begin{itemize}
        \item Jedes Paket erhält eine neue Sequenznummer 
        \item Zur eindeutigen Identifikation der Daten innerhalb eines Streams werden zusätzlich Stream-ID und Stream-Offset verwendet.
        \item Ermöglicht exakte Nachverfolgung von bestätigten und verlorenen Paketen
    \end{itemize}
\end{frame}

\begin{frame}{Ack-Zurücknehmen}
    \begin{itemize}
        \item Bereits bestätigte Pakete können nicht zurückgenommen werden
        \item Vereinfachte Verlustbehandlung auf Sender- und Empfängerseite
    \end{itemize}
\end{frame}

\begin{frame}{Ack-Ranges}
    \begin{itemize}
        \item Mehrere ACK-Ranges in einem Paket möglich
        \item Schnellere Paketverlust-Erkennung
        \item Reduziert unnötige Retransmissions, besonders bei Netzwerken mit hohem Paketverlust
    \end{itemize}
\end{frame}

\begin{frame}{Korrektur für verzögerte ACKs}
    \begin{itemize}
        \item ACKs kodieren die Verzögerung zwischen Paketeingang und ACK-Versand
        \item Ermöglicht genauere RTT-Schätzungen beim Empfänger
    \end{itemize}
\end{frame}

\begin{frame}{Fehlererkennung}
        \textbf{TCP:} Prüfsumme auf Paketebene zur Sicherstellung von Header- und Datenintegrität \\
        \vspace{0.3cm}
        \textbf{QUIC:} 
            \begin{itemize}
                \item Integrierte TLS-1.3-Verschlüsselung für Integrität und Manipulationsschutz
                \item Jedes Paket enthält einen Authentifizierungs-Tag zur Erkennung von Änderungen
                \item Ersetzt klassische Prüfsumme durch kryptografisch gesicherte Integrität
            \end{itemize}
        \textbf{Forward Error Correction:} Standardmäßig nicht vorhanden, aber einfacher nachrüstbar dank Userspace-Implementierung
\end{frame}

\subsection{Features}

\begin{frame}{Head-of-Line Blocking}
    \textbf{Problem bei TCP:} 
        \begin{itemize}
            \item HTTP baut oft mehrere parallele TCP-Verbindungen auf
            \item Verzögerungen oder Paketverluste in einer Verbindung blockieren Daten in anderen Verbindungen
        \end{itemize}

    \vspace{0.3cm}
    \textbf{QUIC-Lösung:}
        \begin{itemize}
            \item Mehrere unabhängige Streams über eine einzige Verbindung
            \item Effektive Vermeidung von Head-of-Line Blocking, schnellerer und effizienterer Datenfluss
        \end{itemize}
\end{frame}

\begin{frame}{Connection Migration}
    QUIC unterstützt \textbf{Connection Migration}:
    \begin{itemize}
        \item Verbindung bleibt bestehen, selbst wenn sich IP-Adresse des Clients ändert
    \end{itemize}
    \vspace{0.3cm}
    Bei TCP würde die Verbindung abbrechen und neu aufgebaut werden müssen\\
    \vspace{0.3cm}
    QUIC ermöglicht dies durch:
    \begin{itemize}
        \item Übertragung über UDP
        \item Eindeutige Identifizierung der Verbindung durch die \textbf{Connection ID} im Header
    \end{itemize}
\end{frame}

\begin{frame}{0-RTT}
    \begin{itemize}
        \item QUIC unterstützt \textbf{0-RTT}:
        \begin{itemize}
            \item Ermöglicht, dass der Client beim Wiederaufbau einer Verbindung sofort Daten senden kann
            \item Der vollständige Handshake muss nicht erneut durchlaufen werden
        \end{itemize}
        \item Verwendung eines speziellen \textbf{0-RTT Long Headers}:
        \begin{itemize}
            \item Identifiziert die Verbindung eindeutig über die Connection-ID
        \end{itemize}
    \end{itemize}
\end{frame}

\begin{frame}{Verschlüsselung}
    \begin{itemize}
        \item QUIC implementiert \textbf{TLS 1.3} direkt im Protokoll
        \item Daten werden verschlüsselt und vor Manipulation geschützt
        \item Ein zusätzlicher Verschlüsselungslayer ist nicht erforderlich
    \end{itemize}
\end{frame}

\begin{frame}{User-Space Implementierung}
    \begin{itemize}
        \item QUIC wird vollständig im \textbf{Userspace} implementiert
        \item Erweiterungen, Updates oder experimentelle Features lassen sich einfacher umsetzen
        \item Kein Eingriff in den Kernel oder bestehende Netzwerk-Stacks notwendig
    \end{itemize}
\end{frame}
